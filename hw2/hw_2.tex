\documentclass[12pt]{article}
\usepackage[margin=1in]{geometry}
\usepackage{amsmath,amssymb,amsthm}
\usepackage{enumitem}

\newcommand{\N}{\mathbb{N}}
\newcommand{\R}{\mathbb{R}}
\newcommand{\Z}{\mathbb{Z}}
\newcommand{\sd}{\mathbin{\triangle}}  % symmetric difference

% Answer environment -- fill in your proofs here
\newenvironment{example}{\begin{proof}[Example]}{\end{proof}}

\title{Mathematics 629 --- Homework 2}
\author{Alex Kim}
\date{Due: Friday, February 13, 2026}

\begin{document}
\maketitle

\begin{enumerate}[label=\textbf{\arabic*.},leftmargin=*,itemsep=12pt]

	%%% Problem 1 %%%%%%%%%%%%%%%%%%%%%%%%%%%%%%%%%%%%%%%%%%%
	\item Give an example of two $\sigma$-algebras whose union is not a $\sigma$-algebra.

	      \begin{example}
		      Let $X = \{1,2,3\}$ be a set with $M_1 = \{\emptyset, \{1\}, \{2,3\}, X\}$ and $M_2 = \{\emptyset, \{2\}, \{1,3\}, X\}$. Notice that

		      $$M_1 \cup M_2 = \{\emptyset, \{1\}, \{0\}, \{2,3\}, \{1,3\}, X\}$$

		      which violates property 3 of the definition of an algebra on $X$ since, for example, $\emptyset^C = M_1 \cup M_2 \notin M_1\cup M_2$.
	      \end{example}

	      %%% Problem 2 %%%%%%%%%%%%%%%%%%%%%%%%%%%%%%%%%%%%%%%%%%%
	\item Prove: An algebra $\mathcal{A}$ is a $\sigma$-algebra if and only if it is closed under countable increasing union (that is, if $E_j$ is a sequence of sets in $\mathcal{A}$ with $E_j \subset E_{j+1}$, then $\bigcup_j E_j \in \mathcal{A}$).

	      \begin{proof}
		      First, assume $\mathcal{A}$ is a $\sigma$-algebra on some set $X$. Since $\{E_j\} \in \mathcal{A}$ is an increaseing sequence of sets for all $j \in \mathbb{N}$, and because $E_{j-1} \subset E_{j}$, it follows that $\bigcup_{j}E_j \in A$ by definition of $\sigma$-algebras. 
		      Next, let $\{E_j\}$ be any sequence of sets in $\mathcal{A}$ and let $F_n = \bigcup_{j = 1}^n E_j$ for some $n \in \mathbb{N}$. 
		      Since $\{E_j\}$ is any sequence, then $F_n \subset F_{n+1}$, thus $\{F_n\}$ is increasing. 
		      By definition of $\mathcal{A}$ as an algebra, $F_n \in A$, and $\bigcup_{n} F_n \in \mathcal{A}$ because $\mathcal{A}$ is assumed to be closed under countable increasing unions. 
		      Thus by definition, $\mathcal{A}$ is a $\sigma$-algebra because $\bigcup_{n} F_n = \bigcup_{j} E_j \in \mathcal{A}$.    
	      \end{proof}

	      %%% Problem 3 %%%%%%%%%%%%%%%%%%%%%%%%%%%%%%%%%%%%%%%%%%%
	\item A family of subsets of $X$ is called a \emph{ring} if the following axioms hold:
	      \begin{itemize}[nosep]
		      \item If $E_1, \dots, E_n \in \mathcal{R}$ then $\bigcup_{j=1}^n E_j \in \mathcal{R}$.
		      \item If $E \in \mathcal{R}$ and $F \in \mathcal{R}$ then $E \setminus F \in \mathcal{R}$.
	      \end{itemize}
	      A ring that is closed under countable unions is called a $\sigma$-ring. Prove:

	      \begin{enumerate}[label=(\roman*),itemsep=6pt]
		      \item Rings are closed under finite intersections. $\sigma$-rings are closed under countable intersections.

		            \begin{proof}
				    Let $E, F \subset \mathcal{R}$ be subsets.
				    Observe that $E\backslash (E \backslash F) = E \cap F \in \mathcal{R} $. Let $\{E_j\}^n$ be a finite collection of sets in $\mathcal{R}$. By the previous observations, $E_1 \cap E_2 \in \mathcal{R}$. Assume 
				    for some $k \geq 2$, $\bigcap_{j=1}^k E_j \in \mathcal{R}$. Then for $E_{k+1} \in \mathcal{R}$, $\bigcap_{j=1}^k E_j \cap E_{k+1} \in \mathcal{R}$ by the base case.
				    \newline
				    Suppose $\mathcal{R}$ is a $\sigma$-ring. Then for $E_1, E_2 \cdots \in \mathcal{R}$, by definition, 
				    $\bigcup_{j=1}^\infty E_j \in \mathcal{R}$. By the previous identity, 
				    the countable intersection of sets in $\mathcal{R}$, $\bigcap{j=1}^\infty E_j = E_1 \backslash \bigcup{j=1}^\infty (E_1 \backslash E_j)$. Since each $E_1\backslash E_j \in \mathcal{R}$ by axiom 2, and $\mathcal{R}$ 
				    is closed under countable unions by definition of $\sigma$-rings, then
				    $\bigcap{j=1}^\infty E_j \in \mathcal{R}$ by axiom 2. 

		            \end{proof}

		      \item A ring is an algebra if and only if $X \in \mathcal{R}$.

		            \begin{proof}
				    First, suppose a ring $\mathcal{R}$ is an algebra. Then by definition, $\mathcal{R}$ is a collection of subsets over some set $X$ where $X \in \mathcal{R}$. Then, suppose $X \in \mathcal{R}$ where $\mathcal{R}$ is a ring. By axiom 1 of rings, $\mathcal{R}$, for all $E, F \in \mathcal{R}
				    $, $E \cup F \in \mathcal{R}$. Since $X \in \mathcal{R}$ from the assumption, then for all $E \in \mathcal{R}$, $X\backslash E \in \mathcal{R}$
				    by axiom 2 of rings. Thus, $\mathcal{R}$ is an algebra by definition.
		            \end{proof}

		      \item If $\mathcal{R}$ is a $\sigma$-ring then the collection $\{E \subset X : E \in \mathcal{R} \text{ or } E^c \in \mathcal{R}\}$ is a $\sigma$-algebra.

		            \begin{proof}
				    Let $\mathcal{R}$ be a $\sigma$-ring and define collection 
				    $C = \{E \subset X : E \in \mathcal{R} \text{ or } X\backslash E
				    \in \mathcal{R}\}$. By axiom 2 of rings, then for all $E \in C$, $X\backslash E \in C$ also. Next, let $E_1, E_2,\cdots, E_n \in C$. Notice that since $E\backslash E = \emptyset
				    \in \mathcal{R}$, so $\emptyset^{c} = X \in C$ by construction. To show $C$ is closed under countable unions, if $\mathcal{R} \in C$, we are done. Otherwise, consider some $E_k \notin \mathcal{R}$. Since $\cap E_n^c \subset E_k^c$, then 
				    $$\bigcap_n E_n^c = E_k^c \backslash (\bigcup_n E_k^c \cap E_n).$$
				    so if $E_n \in \mathcal{R}$, $E_n \cap E_k^c \in \mathcal{R}$ and if 
				    $E_n^c \in \mathcal{R}$, $E_n^c \cap E_k^c \in \mathcal{R}$ also. Thus,
				    $\bigcap E_n^c = \bigcup E_n \in \mathcal{R}$, so $\mathcal{R}$ is a $\sigma$-algebra by definition. 

		            \end{proof}

		      \item If $\mathcal{R}$ is a $\sigma$-ring then the collection $\{E \subset X : E \cap F \in \mathcal{R} \text{ for all } F \in \mathcal{R}\}$ is a $\sigma$-algebra.

		            \begin{proof}
				Let $\mathcal{R}$ be a $\sigma$-ring and define collection $C =\{E \subset X : E \cap F \in \mathcal{R} \text{ for all } F \in \mathcal{R}$. 
				Let $E_1, E_2, \cdots E_n \in \mathcal{R}$ and $F \in \mathcal{R}$. Since $E \cap E = \emptyset \in C$, by axiom 2 of rings
				$X \in \mathcal{R}$, so $X \cap \emptyset = X \in C$ by construction. By a similar argument, $\bigcup_n E_n \in C$ since $\mathcal{R}$ is closed under countable unions. Notice that because $X \in C$, and $X = E \cup (X\backslash E)$ for all $E \in C$, then $X\backslash E \in C$ since $C$ is closed under
				unions. Thus, $C$ is a $\sigma$-algebra by definition.
		            \end{proof}
	      \end{enumerate}

	      %%% Problem 4 %%%%%%%%%%%%%%%%%%%%%%%%%%%%%%%%%%%%%%%%%%%
	\item Let $\mathcal{B}_{\R}$ be the Borel $\sigma$-algebra on the real line.

	      \begin{enumerate}[label=(\roman*),itemsep=6pt]
		      \item Prove that $\mathcal{B}_{\R}$ is generated by the closed and bounded intervals, i.e.\ sets of the form $[a,b]$ with $a,b \in \R$ and $a < b$.

		            \begin{answer}
			            % YOUR ANSWER HERE
		            \end{answer}

		      \item Prove that $\mathcal{B}_{\R}$ is generated by the collection of sets of the form $(-\infty, a)$ with $a \in \R$.

		            \begin{answer}
			            % YOUR ANSWER HERE
		            \end{answer}
	      \end{enumerate}

	      %%% Problem 5 %%%%%%%%%%%%%%%%%%%%%%%%%%%%%%%%%%%%%%%%%%%
	\item Let $(X, \mathcal{M}, \mu)$ be a measure space. Prove that for all $E, F \in \mathcal{M}$,
	      \[
		      \mu(E) + \mu(F) = \mu(E \cup F) + \mu(E \cap F).
	      \]

	      \begin{proof}
		      By decomposing $E$ and $F \in \mathcal{M}$ into $E = (E\backslash F) \uplus (E \cap F)$ and $F = (F\backslash E) \uplus (E \cap F)$ then 
		      $$ E \cup F = (E \backslash F) \uplus (E \cap F) \uplus (E \backslash F).$$
		      Then by subtitution,
		      \begin{align}
			      \mu(E) + \mu(F) &= \mu(E\backslash F) + 2\mu(E \cap F) + \mu(F\backslash E)\\
		      			&=  \mu(E \cup F) + \mu (E \cap F).
			\end{align}
			by the definition of measure.
	      \end{proof}

	      %%% Problem 6 %%%%%%%%%%%%%%%%%%%%%%%%%%%%%%%%%%%%%%%%%%%
	\item Let $(X, \mathcal{M}, \mu)$ be a measure space. The symmetric difference of two sets $E$ and $F$ is given by $E \sd F := (E \setminus F) \cup (F \setminus E)$. Prove:

	      \begin{enumerate}[label=(\roman*),itemsep=6pt]
		      \item If $E, F \in \mathcal{M}$ and $\mu(E \sd F) = 0$ then $\mu(E) = \mu(F)$.

		            \begin{proof}
			            % YOUR ANSWER HERE
		            \end{proof}

		      \item Define a relation on $\mathcal{M}$ by saying that $E \sim F$ if and only if $\mu(E \sd F) = 0$. Show that $\sim$ is an equivalence relation.

		            \begin{proof}
			            % YOUR ANSWER HERE
		            \end{proof}

		      \item For $E, F$ define $\rho(E,F) = \mu(E \sd F)$. Prove a triangle inequality
		            \[
			            \rho(E,G) \le \rho(E,F) + \rho(F,G)
		            \]
		            for all $E, F, G$. Argue that $\rho$ defines a metric on the space $\mathcal{M}/{\sim}$ of equivalence classes.

		            \begin{proof}
			            % YOUR ANSWER HERE
		            \end{proof}
	      \end{enumerate}

\end{enumerate}

\end{document}
