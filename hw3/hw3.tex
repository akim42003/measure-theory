\documentclass[12pt]{article}

% Packages
\usepackage[margin=1in]{geometry}
\usepackage{amsmath}
\usepackage{amssymb}
\usepackage{amsthm}
\usepackage{enumitem}

% Source - https://tex.stackexchange.com/a/21229
% Posted by Carsten Thiel
% Retrieved 2026-02-01, License - CC BY-SA 3.0

\theoremstyle{definition}
\newtheorem{exmp}{Example}


% Title information
\title{Mathematics 629\\
Spring 2026\\
Homework Assignment 3}
\author{Alexander Kim}
\date{Due: 2/20/2026}

\begin{document}

\maketitle

\begin{enumerate}

	\item Prove that the Borel $\sigma$-algebra in the plane, $\mathcal{B}_{\mathbb{R}^2}$, is generated by the collection of sets of the form $V_{a,b} = \{(x_1, x_2) : x_1 \geq a, x_2 > b\}$, $a, b \in \mathbb{R}$.

	      \begin{proof}
		      Since $V_{a,b}$ is constructed such that $(x_1, x_2) \in V_{a,b}$ for $x_1 \geq a, x_2 > b$ given $a,b \in \mathbb{R}$, then $\{[a,\infty); a \in \mathbb{R}\}$ generates $\mathcal{B}_{\mathbb{R}}$ by proposition 1.35. Similarly $\{(b, \infty); b\in \mathbb{R}\}$ generates $\mathcal{B}_{\mathbb{R}}$.
		      Thus, $\mathfrak{M}(V_{a,b}) \supset \mathcal{B}_{\mathbb{R}^2}$ because $V_{a,b} = [a,\infty) \times (b,\infty) \in \mathcal{B}_{\mathbb{R}^2}$ for all $a,b \in \mathbb{R}$.
		      Next, we will show $\mathcal{B}_{\mathbb{R}^2} \subset \mathfrak{M}(V_{a,b})$. By definition of a plane, for all $(a_i, b_i)$ and $(a_j, b_j)$, $(a_i, b_i) \times (a_j, b_j) \in \mathcal{B}_{\mathbb{R}}$.
		      Notice that for $a_i, b_i \in \mathbb{R}$, since $[a_i, \infty)\backslash(b_i,\infty) =
			      [a_i, \infty) \cap (b_i,\infty)^c \in \mathfrak{M}(V_{a,b})$ by definition of $\sigma$-algebras, then $[a_i,\infty)\backslash(b_i,\infty) = (a_i,b_i) \in \mathfrak{M}(V_{a,b}$ also.
		      Thus, $(a_i, b_i)\times(a_j, b_j) \in \mathfrak{M}(V_{a,b})$ which shows
		      $\mathcal{B}_{\mathbb{R}^2} \subset \mathfrak{M}(V_{a,b})$.
	      \end{proof}

	\item Let $E \subset \mathbb{R}$ be a Lebesgue measurable set of positive Lebesgue measure, $m(E) > 0$. Show that for any $\beta < 1$ there is an open interval $I$ such that $m(E \cap I) > \beta \, m(I)$.

	      \textit{Hint: Argue by contradiction and think about the definition of the outer measure that determines Borel-Lebesgue measure.}

	      \begin{proof}
		      % Your proof here.
	      \end{proof}

	\item Let $E \subset \mathbb{R}$ be a set of positive Lebesgue measure. Let $N \in \mathbb{N}$. Show that $E$ contains an arithmetic progression of length $N$, i.e.\ there is an $a > 0$ and a real number $x$ so that $x, x + a, x + 2a, \ldots, x + (N-1)a$ belong to $E$.

	      \textit{Hint: Adapt the proof of Steinhaus' theorem.}

	      \begin{proof}
		      Without loss of generality, let $E \subset [0,1]$. By theorem 1.64, let $E$ be compact such that for $a>0$ there is $U \supset E$ such that $m(U) < (1 + a)m(E)$. Let $\epsilon = \frac{1}{2}\text{dist}(E, U^c) > 0$ and $\vert kt\vert < \frac{\epsilon}{(N-1)}$.
		      Then, for all $x\in E$, $x + kt \in E$ also.
		      We want to show $E\cap (E-t)\cap(E-2t)\cap \cdots \cap (E-Nt) = \emptyset$ by showing the measure of intersections is greater than 0.
		      By the construction of $\epsilon$, $(E-kt) \cap U^c = \emptyset$ if $t < \epsilon$, thus
		      $$ m(U\backslash (E-kt)) = m(U) - m(E-kt) = m(u) - m(E) < (1+a)m(E) - m(E) \leq am(E).$$
		      By subadditivity,
		      $$ m(\bigcup_{k=0}^{N-1} (E-kt)) \geq m(U) - \sum_{k=0}^{N-1} (U\backslash (E-kt))$$
		      From the previous inequality showing $m(U\backslash (E-kt)) \leq am(E)$ for $N$ terms then,
		      $$m(\bigcup_{k=1}^{N-1}(E-kt)) \geq m(U) - Nam(E) \geq (N-1)m(E) > 0.$$
		      if we specify that $a < \frac{1}{N-1}$.


	      \end{proof}

	\item Let $E$ be the set of all real numbers which have the digit 7 missing in their decimal expansion. Show that $E$ is a Lebesgue null set.

	      \begin{proof}
		      Let $S = [0,1)$. Then it suffices to show $\mu(E\cap S) = 0 <\epsilon$ for all $\epilon > 0$ since $\mu(E) = \mu(\bigcup_{i\in \mathbb{Z}} (S + i))$ by countable additivity. Next, let $P_0$ be a partition of $[0,1)$ containing sets of the form
		      $[\frac{k}{10}, \frac{k+1}{10})$ for $0\leq k \leq 9$. Since $\mu(P_0) = \frac{9}{10}$, notice that forming a similar partition $P_j$ on
		      every $p \in P_0$ has measure $\frac{9}{10} \mu(p)$ so $$\mu(\bigcup_{j=0}^9 P_j) = (\frac{9}{10})^2.$$
		      Then for $n\in \mathbb{N}$ decimal places of $x \in S$,
		      $$\mu(E) = (\frac{9}{10})^n \rightarrow 0$$
		      as a geometric sequence.
	      \end{proof}

\end{enumerate}

\end{document}
