\documentclass[12pt]{article}

% Packages
\usepackage[margin=1in]{geometry}
\usepackage{amsmath}
\usepackage{amssymb}
\usepackage{amsthm}
\usepackage{enumitem}

% Source - https://tex.stackexchange.com/a/21229
% Posted by Carsten Thiel
% Retrieved 2026-02-01, License - CC BY-SA 3.0

\theoremstyle{definition}
\newtheorem{exmp}{Example}


% Title information
\title{Mathematics 629\\
Spring 2026\\
Homework Assignment 3}
\author{Alexander Kim}
\date{Due: 2/20/2026}

\begin{document}

\maketitle

\begin{enumerate}

	\item Prove that the Borel $\sigma$-algebra in the plane, $\mathcal{B}_{\mathbb{R}^2}$, is generated by the collection of sets of the form $V_{a,b} = \{(x_1, x_2) : x_1 \geq a, x_2 > b\}$, $a, b \in \mathbb{R}$.

	      \begin{proof}
		      % Your proof here.
	      \end{proof}

	\item Let $E \subset \mathbb{R}$ be a Lebesgue measurable set of positive Lebesgue measure, $m(E) > 0$. Show that for any $\beta < 1$ there is an open interval $I$ such that $m(E \cap I) > \beta \, m(I)$.

	      \textit{Hint: Argue by contradiction and think about the definition of the outer measure that determines Borel-Lebesgue measure.}

	      \begin{proof}
		      % Your proof here.
	      \end{proof}

	\item Let $E \subset \mathbb{R}$ be a set of positive Lebesgue measure. Let $N \in \mathbb{N}$. Show that $E$ contains an arithmetic progression of length $N$, i.e.\ there is an $a > 0$ and a real number $x$ so that $x, x + a, x + 2a, \ldots, x + (N-1)a$ belong to $E$.

	      \textit{Hint: Adapt the proof of Steinhaus' theorem.}

	      \begin{proof}
		      % Your proof here.
	      \end{proof}

	\item Let $E$ be the set of all real numbers which have the digit 7 missing in their decimal expansion. Show that $E$ is a Lebesgue null set.

	      \begin{proof}
		      % Your proof here.
	      \end{proof}

\end{enumerate}

\end{document}
