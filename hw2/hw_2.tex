\documentclass[12pt]{article}
\usepackage[margin=1in]{geometry}
\usepackage{amsmath,amssymb,amsthm}
\usepackage{enumitem}

\newcommand{\N}{\mathbb{N}}
\newcommand{\R}{\mathbb{R}}
\newcommand{\Z}{\mathbb{Z}}
\newcommand{\sd}{\mathbin{\triangle}}  % symmetric difference

% Answer environment -- fill in your proofs here
\newenvironment{example}{\begin{proof}[Example]}{\end{proof}}

\title{Mathematics 629 --- Homework 2}
\author{Alex Kim}
\date{Due: Friday, February 13, 2026}

\begin{document}
\maketitle

\begin{enumerate}[label=\textbf{\arabic*.},leftmargin=*,itemsep=12pt]

	%%% Problem 1 %%%%%%%%%%%%%%%%%%%%%%%%%%%%%%%%%%%%%%%%%%%
	\item Give an example of two $\sigma$-algebras whose union is not a $\sigma$-algebra.

	      \begin{example}
		      Let $X = \{1,2,3\}$ be a set with $M_1 = \{\emptyset, \{1\}, \{2,3\}, X\}$ and $M_2 = \{\emptyset, \{2\}, \{1,3\}, X\}$. Notice that

		      $$M_1 \cup M_2 = \{\emptyset, \{1\}, \{0\}, \{2,3\}, \{1,3\}, X\}$$

		      which violates property 3 of the definition of an algebra on $X$ since, for example, $\emptyset^C = M_1 \cup M_2 \notin M_1\cup M_2$.
	      \end{example}

	      %%% Problem 2 %%%%%%%%%%%%%%%%%%%%%%%%%%%%%%%%%%%%%%%%%%%
	\item Prove: An algebra $\mathcal{A}$ is a $\sigma$-algebra if and only if it is closed under countable increasing union (that is, if $E_j$ is a sequence of sets in $\mathcal{A}$ with $E_j \subset E_{j+1}$, then $\bigcup_j E_j \in \mathcal{A}$).

	      \begin{proof}
		      First, assume $\mathcal{A}$ is a $\sigma$-algebra on some set $X$. Since $\{E_j\} \in \mathcal{A}$ is an increaseing sequence of sets for all $j \in \mathbb{N}$, and because $E_{j-1} \subset E_{j}$, it follows that $\bigcup_{j}E_j \in A$ by definition of unions. Next, let $\bigcup_j E_j \in \mathcal{A}$ and $\{E_j\} \in \mathcal{A}$ is a countable increasing union of sets.
	      \end{proof}

	      %%% Problem 3 %%%%%%%%%%%%%%%%%%%%%%%%%%%%%%%%%%%%%%%%%%%
	\item A family of subsets of $X$ is called a \emph{ring} if the following axioms hold:
	      \begin{itemize}[nosep]
		      \item If $E_1, \dots, E_n \in \mathcal{R}$ then $\bigcup_{j=1}^n E_j \in \mathcal{R}$.
		      \item If $E \in \mathcal{R}$ and $F \in \mathcal{R}$ then $E \setminus F \in \mathcal{R}$.
	      \end{itemize}
	      A ring that is closed under countable unions is called a $\sigma$-ring. Prove:

	      \begin{enumerate}[label=(\roman*),itemsep=6pt]
		      \item Rings are closed under finite intersections. $\sigma$-rings are closed under countable intersections.

		            \begin{proof}
			            % YOUR ANSWER HERE
		            \end{proof}

		      \item A ring is an algebra if and only if $X \in \mathcal{R}$.

		            \begin{proof}
			            % YOUR ANSWER HERE
		            \end{proof}

		      \item If $\mathcal{R}$ is a $\sigma$-ring then the collection $\{E \subset X : E \in \mathcal{R} \text{ or } E^c \in \mathcal{R}\}$ is a $\sigma$-algebra.

		            \begin{proof}
			            % YOUR ANSWER HERE
		            \end{proof}

		      \item If $\mathcal{R}$ is a $\sigma$-ring then the collection $\{E \subset X : E \cap F \in \mathcal{R} \text{ for all } F \in \mathcal{R}\}$ is a $\sigma$-algebra.

		            \begin{proof}
			            % YOUR ANSWER HERE
		            \end{proof}
	      \end{enumerate}

	      %%% Problem 4 %%%%%%%%%%%%%%%%%%%%%%%%%%%%%%%%%%%%%%%%%%%
	\item Let $\mathcal{B}_{\R}$ be the Borel $\sigma$-algebra on the real line.

	      \begin{enumerate}[label=(\roman*),itemsep=6pt]
		      \item Prove that $\mathcal{B}_{\R}$ is generated by the closed and bounded intervals, i.e.\ sets of the form $[a,b]$ with $a,b \in \R$ and $a < b$.

		            \begin{answer}
			            % YOUR ANSWER HERE
		            \end{answer}

		      \item Prove that $\mathcal{B}_{\R}$ is generated by the collection of sets of the form $(-\infty, a)$ with $a \in \R$.

		            \begin{answer}
			            % YOUR ANSWER HERE
		            \end{answer}
	      \end{enumerate}

	      %%% Problem 5 %%%%%%%%%%%%%%%%%%%%%%%%%%%%%%%%%%%%%%%%%%%
	\item Let $(X, \mathcal{M}, \mu)$ be a measure space. Prove that for all $E, F \in \mathcal{M}$,
	      \[
		      \mu(E) + \mu(F) = \mu(E \cup F) + \mu(E \cap F).
	      \]

	      \begin{proof}
		      % YOUR ANSWER HERE
	      \end{proof}

	      %%% Problem 6 %%%%%%%%%%%%%%%%%%%%%%%%%%%%%%%%%%%%%%%%%%%
	\item Let $(X, \mathcal{M}, \mu)$ be a measure space. The symmetric difference of two sets $E$ and $F$ is given by $E \sd F := (E \setminus F) \cup (F \setminus E)$. Prove:

	      \begin{enumerate}[label=(\roman*),itemsep=6pt]
		      \item If $E, F \in \mathcal{M}$ and $\mu(E \sd F) = 0$ then $\mu(E) = \mu(F)$.

		            \begin{proof}
			            % YOUR ANSWER HERE
		            \end{proof}

		      \item Define a relation on $\mathcal{M}$ by saying that $E \sim F$ if and only if $\mu(E \sd F) = 0$. Show that $\sim$ is an equivalence relation.

		            \begin{proof}
			            % YOUR ANSWER HERE
		            \end{proof}

		      \item For $E, F$ define $\rho(E,F) = \mu(E \sd F)$. Prove a triangle inequality
		            \[
			            \rho(E,G) \le \rho(E,F) + \rho(F,G)
		            \]
		            for all $E, F, G$. Argue that $\rho$ defines a metric on the space $\mathcal{M}/{\sim}$ of equivalence classes.

		            \begin{proof}
			            % YOUR ANSWER HERE
		            \end{proof}
	      \end{enumerate}

\end{enumerate}

\end{document}
